% !TeX spellcheck = en_US
\documentclass[parskip=full]{report}

\nocite{*}

\usepackage{amsmath}
\usepackage{listings}
%\usepackage{beramono}
\usepackage{float}
\usepackage[utf8]{inputenc}
\usepackage[T1]{fontenc}
\usepackage{xcolor}
\usepackage[a4paper, margin={3cm}]{geometry}
\usepackage{hyperref}
\usepackage{graphicx}
\usepackage{svg}
\usepackage{subcaption}
\usepackage{float}
\usepackage{pdfpages}
\usepackage{longtable}
\usepackage{csvsimple}

\usepackage{hyphenat}
\usepackage[english]{babel}
% Lingua per la bibliografia
\usepackage[fixlanguage]{babelbib}
% Carattere monospaziato di default
\renewcommand{\ttdefault}{pcr}

% Stile del codice
\lstdefinestyle{codeStyle}{
	% Colore dei commenti
	commentstyle=\color{teal},
	% Colore delle keyword
	keywordstyle=\color{magenta},
	% Stile dei numeri di riga
	numberstyle=\tiny\color{gray},
	% Colore delle stringhe
	stringstyle=\color{teal},
	% Dimensione e stile del testo
	basicstyle=\ttfamily\scriptsize,
	% newline solo ai whitespaces
	breakatwhitespace=false,     
	% newline si/no
	breaklines=true,                 
	% Posizione della caption, top/bottom 
	captionpos=b,                    
	% Mantiene gli spazi nel codice, utile per l'indentazione
	keepspaces=true,                 
	% Dove visualizzare i numeri di linea
	numbers=left,                    
	% Distanza tra i numeri di linea
	numbersep=5pt,                  
	% Mostra gli spazi bianchi o meno
	showspaces=false,                
	% Mostra gli spazi bianchi nelle stringhe
	showstringspaces=false,
	% Mostra i tab
	showtabs=false,
	% Dimensione dei tab
	tabsize=2,
	% accenti
	literate={á}{{\'a}}1 {à}{{\`a}}1 {é}{{\'e}}1 {è}{{\`e}}1,
	% wrap long lines on new line
	postbreak=\mbox{\textcolor{red}{$\hookrightarrow$}\space},
} \lstset{style=codeStyle}


% Title Page
\title{
	\includegraphics[width=0.333\textwidth]{assets/unipi1.png} \\
	\vspace{4cm}
	Emotion Recognition
}

\author{
	\begin{tabular}{lr}
		Dario Pagani & 585281 \\
		Alessandro Versari & ??????
	\end{tabular}
}


\begin{document}
\maketitle
\tableofcontents

%%%%%%%%%%%%%%%%%%%%%%%%%%%% INIZIO QUA %%%%%%%%%%%%%%%%%%%%

\chapter{Introduction}

\paragraph{}
The intricate interplay of emotions is a quintessential facet of human existence, shaping our experiences, decisions, and interactions. Emotion recognition, a branch of affective computing, has emerged as a pivotal field within artificial intelligence and machine learning, aiming to decipher the complex nuances of human emotions from various data modalities. In recent years, significant advancements in the realm of deep learning and convolutional neural networks (CNNs) have substantially improved the accuracy and robustness of emotion recognition systems. This report embarks on an academic exploration of the integration of Mel-frequency cepstral coefficients (MEL spectrograms), image recognition techniques, and CNN architectures to develop an innovative approach towards emotion recognition.

Emotions, as multi-dimensional constructs, present a multifarious challenge for computational systems seeking to emulate human emotional intelligence. The ability to accurately perceive and classify emotional states from diverse data sources is paramount, as it can have profound implications across a spectrum of applications, including human-computer interaction, mental health monitoring, and personalized user experiences. Recognizing emotions through MEL spectrograms, images, and CNNs represents a pioneering approach that harnesses the synergistic potential of these components to improve the overall efficacy and versatility of emotion recognition systems.

\paragraph{Spectrograms}
Mel-frequency cepstral coefficients, commonly known as MEL spectrograms, offer a unique representation of acoustic signals in the frequency domain, capturing the perceptually relevant features of audio data. Their adoption in emotion recognition is grounded in their capacity to extract discriminative information from speech signals, enabling the identification of subtle variations in vocal expressions that signify emotional states. This report delves into the fundamental principles and techniques underpinning MEL spectrogram extraction and elaborates on their role in enhancing the performance of emotion recognition models.

\paragraph{Image recognition}
Central to the amalgamation of MEL spectrograms and image recognition is the utilization of convolutional neural networks (CNNs), a class of deep learning models renowned for their prowess in pattern recognition and feature extraction from multi-dimensional data. This report underscores the pivotal role of CNNs as the backbone of the proposed architecture, expounding upon their architectural nuances and training methodologies specific to emotion recognition. An in-depth exploration of transfer learning techniques and fine-tuning strategies is also undertaken to elucidate the ways in which pre-trained CNN models can be adapted for the task of emotion recognition.
\chapter{Dataset}

\section{Sources}

\paragraph{}
We combined multiple dataset to create our voices' corpus:

\begin{itemize}
	\item 
	\textbf{Emo-DB} \cite{krautdb} 800 recording spoken by 10 actors (5 males and 5 females); 7 emotions: anger, neutral, fear, boredom, happiness, sadness, disgust
	
	\item
	\textbf{EMOVO} \cite{costantini-etal-2014-emovo} 6 actors who played 14 sentences; 6 emotions: disgust, fear, anger, joy, surprise, sadness
	
	\item 
	\textbf{Emov-DB} \cite{adigwe2018emotional} Recordings for 4 speakers- 2 males and 2 females; The emotional styles are neutral, sleepiness, anger, disgust and amused
	
	\item 
	\textbf{JL corpus} \cite{jl-corpus} 2400 recording of 240 sentences by 4 actors (2 males and 2 females); 5 primary emotions: angry, sad, neutral, happy, excited. 5 secondary emotions: anxious, apologetic, pensive, worried, enthusiastic
	
	\item 
	\textbf{Multimodal EmotionLines Dataset (MELD)} \cite{poria2019meld} has been created by enhancing and extending EmotionLines dataset. MELD contains the same dialogue instances available in EmotionLines, but it also encompasses audio and visual modality along with text. MELD has more than 1400 dialogues and 13000 utterances from Friends TV series. Each utterance in a dialogue has been labeled with— Anger, Disgust, Sadness, Joy, Neutral, Surprise and Fear
	
	\item
	The \textbf{Ryerson Audio-Visual Database of Emotional Speech and Song (RAVDESS)} \cite{livingstone_steven_r_2018_1188976} contains 7356 files (total size: 24.8 GB). The database contains 24 professional actors (12 female, 12 male), vocalizing two lexically-matched statements in a neutral North American accent. Speech includes calm, happy, sad, angry, fearful, surprise, and disgust expressions, and song contains calm, happy, sad, angry, and fearful emotions. We used only speeches
\end{itemize}

\section{Preprocessing}

\paragraph{Formats}
We were presented with a variety of audio formats, such as WAW and OGG, from those we created the MEL spectrograms with \texttt{librosa}'s help, a Python library to handle audio signals from heterogeneous sources

\paragraph{Spectrograms}
Each audio sample was loaded with a sampling rate of 44100 Hertz, file which had a different sampling rate were automatically resampled by the library; then each sample was divided, when possible, in multiple segments of 1.75 seconds of length each; finally spectrograms were computed and saved to single channel PNG file.

\paragraph{Classes}
Each PNG was saved in a sub-directory named after its class.

\paragraph{Multiprocessing}
Python's multiprocessing capabilities were used to speed-up the computation.

\paragraph{Procedure and example}
Here it follows the procedure used to generate the images:

\lstinputlisting[language=Python, linerange={20-89}]{../preprocess.py}

\begin{figure}[H]
	\centering
	\includegraphics[width=0.7\linewidth, angle=90, interpolate=false]{assets/ex1.big.png}
	\caption{Example of spectrogram of length 1.75 seconds}
	\label{fig:ex2}
\end{figure}

\paragraph{Voice extraction}
We tried to isolate human voice from other background noise through librosa's buil-in methods but results were unsatisfactory as the accuracy and F1-score dropped significantly, thus we used the raw audios without any cleaning --- i.e. we set \texttt{ISOLATE\_VOICE = False} in the generation program.

It's certainly possible to improve models' performances with proper audio cleaning technique but we lack both the skill and the time to dive deeper into the manner. For example, we could have tried to cut frequencies that are not used by humans' voices.

\section{Classes}

\paragraph{}
We obtained the following dataset:

\vspace{10mm}
\begin{tabular}{|c|r|r|}
	\hline
	Class & Cardinality & Ratio \\
	\hline\hline
	SURPRISE	& 1063 & 9.51\% \\
	\hline
	SADNESS		& 1132 & 10.13\% \\
	\hline
	NEUTRAL		& 2811 & 25.15\%\\
	\hline
	HAPPINESS	& 2807 & 25.15\%\\
	\hline
	DISGUST		& 1204 & 10.77\%\\
	\hline
	ANGER		& 2160 & 19.33\%\\
	\hline\hline
	\textbf{Total} & 11177 & 100\% \\
	\hline
\end{tabular}

\paragraph{}
Pictures are 128x151 in size.


\chapter{Training}

\section{Network from scratch}

\paragraph{Modus Operandi}
We used the \emph{trial-and-error} approach to create the networks discussed in this chapter, since there are too many hyper-parameters to try.  We started with a small model that did not fit well and then gradually increase
its size until it started to generalize. We also tried to augment the input data by playing with the pictures' contrast in a model but with unimpressive results.

\paragraph{General Architecture}
We opted for a batch size of 32, representing the number of samples considered in each iteration. Following the input layer, we applied normalization using a Rescaling Layer to ensure that each pixel value falls within the $[0,1]$ range. Our choice then led us to employ four convolutional layers, incorporating zero-padding to grant equal significance to every pixel within the image. This approach is valuable because even the border regions hold relevance, despite our efforts to crop the images as extensively as possible. We kept the stride at its default value of 1 to avoid any adverse impact on accuracy. Our activation functions of choice were ReLU, defined as $f(x) = \max\{0, x\}$. For the size of the local receptive fields, we stuck with the default 3x3 dimensions.

In the initial convolutional layer, we utilized 32 filters, with a doubling of filter count for each subsequent convolutional layer. To enhance feature extraction, we applied max-pooling after each convolutional layer, thereby consolidating small regions into a single value.

\subsection{Two dense layers}

\paragraph{Architecture}
We used two dense layers with 128 and 256 neurons:

\lstinputlisting[language=Python,linerange={218-233}]{../train_scratch.py}

We tried a low drop-out to fight against overfitting.

\paragraph{Results}

\subsection{Three dense layers}

\subsection{Data augmentation}

\paragraph{Architecture}

\paragraph{Augmentation}
We used a layer \texttt{tf.keras.layers.RandomContrast(0.5)} to adjust the spectrograms' contrasts during training

\paragraph{Result}

\section{Pre-trained models}

\subsection{Feature extraction from VGG16}

\subsection{Fine tuning with VGG16}
\chapter{Conclusions}

\section{Literature}

We found an article \cite{womansarticle} were the author worked on a subset of our dataset and she obtained an accuracy of 78\%. She obtained such result with just three dense layers:

\begin{lstlisting}[language=Python]
x = vgg_model.output
x = Flatten()(x) # Flatten dimensions to for use in FC layers
x = Dense(512, activation='relu')(x)
x = Dropout(0.5)(x) # Dropout layer to reduce overfitting
x = Dense(256, activation='relu')(x)
x = Dense(8, activation='softmax')(x) # Softmax for multiclass
transfer_model = Model(inputs=vgg_model.input, outputs=x)	
\end{lstlisting}

while reaching almost 100\% accuracy on the training set, this might be due to \emph{overfit} over the data.

\section{Model selection}





\bibliographystyle{plain} % We choose the "plain" reference style
\bibliography{refs.bib} % Entries are in the refs.bib file

\end{document}          
